\documentclass{article}
\usepackage{tikz}
\usetikzlibrary{calc}

\usepackage{amsmath}   

\newcommand{\tikzmark}[1]{%
\tikz[overlay,remember picture] \node (#1) {};
}
% stile per la creazione della freccia 
\tikzset{square arrow/.style={to path={-- ++(0,.25) -| (\tikztotarget)}}}
\thispagestyle{empty}
\begin{document}
\[
=\, -\nu_{gx}\,  \dfrac{\partial \, \left<\tikzmark{a}n\right>}{\partial\,x} \, =\, -\nu_{gx} \, \dfrac{\partial \left<\tikzmark{b}\eta_{0}\right>}{\partial \,x} \, \dfrac{\partial \,T}{\partial \,x}
\]
\tikz[overlay,remember picture]{%
    \draw[<->,square arrow, red, thick] 
        ($(a.north)+(0.25em,0.30ex)$) to ($(b.north)+(0.50em,0.30ex)$) ;
    \node [above, yshift=1.50ex, blue] at  ($(a.north)!0.5!(b.north)$) {label};
}
\end{document}
