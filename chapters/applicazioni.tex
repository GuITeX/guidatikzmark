\chapter{Possibili applicazioni}
\label{chap:applicazioni}
Prima di prendere in esame alcune applicazioni possibili, si illustra brevemente l'utilizzo più semplice possibile.

%--------------------------------------------
\section{L'utilizzo semplice}

%--------------------------------------------
Nella sua prima versione, il codice della \cs{tikzmark} macro si presentava così:
\begin{lstlisting}[frame=lines]
\newcommand{\tikzmark}[1]{
  \tikz[remember picture,overlay]\node (#1) {};
}
\end{lstlisting}
Si ipotizzi di voler creare un collegamento fra due righe di testo con una freccia curva. Attraverso la \cs{tikzmark} macro definita in precedenza si può procedere in questo modo:
\begin{lstlisting}[frame=lines]
\documentclass{article}
\usepackage{tikz}
\newcommand{\tikzmark}[1]{
  \tikz[remember picture,overlay]\node (#1) {};
}

\begin{document}
\tikzmark{testo} Testo \\[2ex]
\tikzmark{altro testo} Altro testo 
\begin{tikzpicture}[remember picture,overlay]
\path[->] (testo)edge[bend right](altro testo);
\end{tikzpicture}
\end{document}
\end{lstlisting}
In primo luogo si posizionano i marcatori vicino alle righe di testo con opportune label; successivamente, in un ambiente \amb{tikzpicture} si realizza il collegamento in modo opportuno partendo proprio dai marcatori. Ci si ricordi di compilare due volte! I risultati della prima e della seconda compilazione sono riportati nelle figure \ref{fig:primacompilazionesemplice} e \ref{fig:secondacompilazionesemplice}.

\begin{figure}
\tikzset{document color/.style={draw,top color=orange!5, bottom color=orange!20}}
\tikzset{document size/.style={minimum width=4cm, minimum height=3cm,
 corner size=20pt,
  inner corner size=10pt,
  corner start=30pt,
  corner end=10pt,
  corner up height=10pt,
  corner down height=30pt,
  fine sep start=3pt,
  fine sep end=3pt,
  }
}
\centering
\subfloat[Prima compilazione\label{fig:primacompilazionesemplice}]{%
  \includedocument[document color,document size]{sorgcrop/img/semplice1-crop}}\quad\quad
\subfloat[Seconda compilazione\label{fig:secondacompilazionesemplice}]{%
 \includedocument[document color,document size]{sorgcrop/semplice-crop}}
\caption{Collegamento fra righe di testo}
\end{figure}

%--------------------------------------------
\section{Riquadri esplicativi}

%--------------------------------------------
In questo esempio si vuole far comparire in una presentazione dei riquadri esplicativi in corrispondenza di determinate parole. 

Siccome \Tikz{} mette a disposizione la libreria \library{shapes.callouts} per questo tipo di forme ed interagisce molto facilmente con Beamer per le animazioni ecco come si può sfruttare in modo intelligente il metodo della \cs{tikzmark} macro. L'identificazione dei termini da spiegare è chiaramente il compito da far svolgere alla nostra \cs{tikzmark} macro che quindi \emph{contrassegnerà} con una label identificativa le coordinate del punto immediatamente successivo alla fine del termine. Successivamente, le label verranno date in pasto ad un comando apposito, definito con \Tikz{}, che creerà effettivamente il riquadro.

Si analizzano ora nel dettaglio i comandi necessari. Per prima cosa la \cs{tikzmark} macro:
\begin{lstlisting}[frame=lines]
\newcommand{\tikzmark}[1]{
  \tikz[overlay,remember picture,xshift=-2pt,baseline=1.5pt] 
  \node (#1) {};
}
\end{lstlisting}
Rispetto al caso precedente ci sono due opzioni aggiuntive: \chiavestyle{xshift=-2pt} e \chiavestyle{baseline=1.5pt}. La loro funzione è semplicemente quella di \emph{spostare} automaticamente il marcatore in modo tale da renderlo più vicino e con un allineamento verticale centrale rispetto al termine da spiegare.

Il seguente codice, invece, crea semplicemente uno ``stile'' \Tikz{}, \chiavestyle{visible on}, utile per rendere visibili gli oggetti in determinati momenti (\emph{overlay specifications} nel manuale di Beamer). 
\begin{lstlisting}[frame=lines]
\tikzset{
    invisible/.style={opacity=0,text opacity=0},
    visible on/.style={alt=#1{}{invisible}},
    alt/.code args={<#1>#2#3}{%
      \alt<#1>{\pgfkeysalso{#2}}{\pgfkeysalso{#3}}
    },
}
\end{lstlisting}
Infine, ecco il comando che materialmente realizza il riquadro esplicativo: si noti che richiede il pacchetto \packstyle{xparse}.
\begin{lstlisting}[frame=lines]
\NewDocumentCommand{\riquadro}{r<> O{opacity=0.8,text opacity=1} m m}
{%
  \tikz[remember picture, overlay]\node[align=justify, fill=cyan!20, 
  text width=4.5cm, #2,visible on=<#1>, rounded corners,
  draw,rectangle callout,anchor=pointer,
  callout relative pointer={(230:1cm)}]
  at (#3) {#4};
}
\end{lstlisting}
In sostanza, \cs{riquadro} richiede 4 argomenti:
\begin{itemize}
\item il primo è obbligatorio: serve per definire in quali momenti della presentazione il riquadro è visibile e utilizza come delimitatori \texttt{< >} in conformità con lo stile adottato da Beamer;
\item il secondo è opzionale: serve per definire la trasparenza del riquadro (si noti che il default prevede una leggera trasparenza in modo tale che il testo in background sia comunque visibile);
\item il terzo è obbligatorio e rappresenta il punto di origine del riquadro, ovvero il punto che è stato contrassegnato con la \cs{tikzmark} macro;
\item il quarto è obbligatorio: è il testo che comparirà nel riquadro.
\end{itemize}
\pagebreak
Ecco il codice con un esempio completo:
\begin{lstlisting}[frame=lines]
\documentclass[10pt]{beamer}
\usepackage[T1]{fontenc}
\usepackage[utf8]{inputenc}
\usepackage{lmodern}
\usepackage{tikz}
\usetikzlibrary{shapes.callouts}

\usepackage{xparse}

\tikzset{
    invisible/.style={opacity=0,text opacity=0},
    visible on/.style={alt=#1{}{invisible}},
    alt/.code args={<#1>#2#3}{%
      \alt<#1>{\pgfkeysalso{#2}}{\pgfkeysalso{#3}}
    },
}

\NewDocumentCommand{\riquadro}{r<> O{opacity=0.8,text opacity=1} m m}
{%
  \tikz[remember picture, overlay]\node[align=justify, fill=cyan!20, 
  text width=4.5cm, #2,visible on=<#1>, rounded corners,
  draw,rectangle callout,anchor=pointer,
  callout relative pointer={(230:1cm)}]
  at (#3) {#4};
}

\newcommand{\tikzmark}[1]{
  \tikz[overlay,remember picture,xshift=-2pt,baseline=1.5pt] 
  \node (#1) {};
}

\begin{document}
\begin{frame}{Esempio}
In una terra lontana, dietro le montagne\tikzmark{montagne} Parole, 
lontani dalle terre di Vocalia e Consonantia, vivono i testi casuali. 
Vivono isolati nella cittadina di Lettere, sulle coste del Semantico, 
un immenso oceano linguistico. Un piccolo ruscello\tikzmark{ruscello} 
chiamato Devoto Oli attraversa quei luoghi, rifornendoli 
di tutte le regolalie di cui hanno bisogno. È una terra paradismatica, 
un paese della cuccagna\tikzmark{cuccagna} in cui golose porzioni 
di proposizioni arrostite volano in bocca a chi le desideri. 
Non una volta i testi casuali sono stati dominati dall'onnipotente 
Interpunzione, una vita davvero non ortografica.
\riquadro<2>{montagne}{Una montagna è un rilievo della superficie 
terrestre che si estende sopra il terreno circostante 
in un'area limitata.}
\riquadro<3>{ruscello}{In idrografia viene chiamato ruscello o rio 
un piccolo corso d'acqua e che confluisce in un corso d'acqua 
maggiore come dimensione e quantità d'acqua trasportata.}
\riquadro<4>[opacity=1]{cuccagna}{Il paese di Cuccagna è un luogo 
ideale, ricordato in molti testi di ogni epoca, nel quale il 
benessere, l'abbondanza e il piacere  è a portata di tutti.}
\end{frame}
\end{document}
\end{lstlisting}

Il risultato viene mostrato nelle figure \ref{fig:primadiapocallout}, \ref{fig:secondadiapocallout}, \ref{fig:terzadiapocallout} e \ref{fig:quartadiapocallout}. Si noti come, in figura \ref{fig:quartadiapocallout}, il riquadro sia completamente opaco e copra il testo: infatti l'ultimo riquadro viene creato con l'opzione \chiavestyle{opacity=1} che sovrascrive l'impostazione di default \chiavestyle{opacity=0.8,text opacity=1}.

\begin{figure}[h]
\tikzset{document color/.style={draw,top color=orange!5, bottom color=orange!20}}
\tikzset{document size/.style={minimum width=5.75cm, minimum height=5.25cm,
 corner size=20pt,
  inner corner size=10pt,
  corner start=30pt,
  corner end=10pt,
  corner up height=10pt,
  corner down height=30pt,
  fine sep start=3pt,
  fine sep end=3pt,
  }
}
\centering
\subfloat[Prima diapositiva\label{fig:primadiapocallout}]{%
  \includedocument[document color,document size][width=0.4\textwidth,keepaspectratio,page=1]{sorgtikz/callout}}\quad\quad
\subfloat[Seconda diapositiva\label{fig:secondadiapocallout}]{%
 \includedocument[document color,document size][width=0.4\textwidth,keepaspectratio,page=2]{sorgtikz/callout}}\\
\subfloat[Terza diapositiva\label{fig:terzadiapocallout}]{%
  \includedocument[document color,document size][width=0.4\textwidth,keepaspectratio,page=3]{sorgtikz/callout}}\quad\quad
\subfloat[Quarta diapositiva\label{fig:quartadiapocallout}]{%
 \includedocument[document color,document size][width=0.4\textwidth,keepaspectratio,page=4]{sorgtikz/callout}}\\ 
\caption{Quadro con riquadri esplicativi}
\end{figure}

%--------------------------------------------
\section{Collegamenti in figure}

%--------------------------------------------
A volte, addirittura all'interno di una figura realizzata con \Tikz{} può essere comodo posizionare dei marcatori per collegare le varie parti. Ecco un esempio in cui, partendo dalla seguente figura di riferimento
\begin{center}
\begin{tikzpicture}[remember picture]
\node[fnode] (r1) at (0,0) {Puntatore \npd Chiave  \npt Testo \npq Puntatori \npc Testo};  
\end{tikzpicture}
\end{center}
si posizioneranno dei marcatori (cerchi di colore nero) vicino ai campi ``Puntatore'' e ``Puntatori''.

La figura di rifermento richiede nel preambolo:
\begin{lstlisting}[frame=lines]
\usetikzlibrary{shapes.multipart,calc}
\newcommand{\npd}{\nodepart{two}}
\newcommand{\npt}{\nodepart{three}}
\newcommand{\npq}{\nodepart{four}}
\newcommand{\npc}{\nodepart{five}}

\tikzset{fnode/.style={rectangle split, 
  rectangle split part align={left,left,left,center,left}, 
  rectangle split parts=5, draw, minimum width =2.75cm,
  rounded corners}
}
\end{lstlisting}
I marcatori vengono definiti, questa volta, come cerchi di colore nero di cui l'utente può personalizzare la dimensione e la distanza rispetto al testo:
\begin{lstlisting}[frame=lines]
\NewDocumentCommand{\anchormark}{O{0.15 cm} m O{0.05}}{%
  \tikz[overlay,remember picture,baseline=-0.5ex,xshift=#1] 
    \node[draw,fill=black,circle,scale=#3] (#2) {};
}
\end{lstlisting}
Anche in questo caso è necessario il pacchetto \packstyle{xparse} e sono 3 gli argomenti richiesti da \cs{anchormark}:
\begin{itemize}
\item il primo è opzionale e rappresenta la distanza fra marcatore e testo: il valore di default è \chiavestyle{0.15cm}, ma si può personalizzare la posizione, spostando ad esempio il marcatore a sinistra del testo inserendo un valore negativo;
\item il secondo, obbligatorio, è la label identificativa con cui è contraddistinto il marcatore;
\item il terzo, nuovamente opzionale, rappresenta il fattore di scala del raggio del cerchio; il valore di default è \chiavestyle{0.05}.
\end{itemize}

Il seguente codice riporta un esempio completo:
\begin{lstlisting}[frame=lines]
\documentclass{article}
\usepackage{xparse}
\usepackage{tikz}

\usetikzlibrary{shapes.multipart,calc}
\newcommand{\npd}{\nodepart{two}}
\newcommand{\npt}{\nodepart{three}}
\newcommand{\npq}{\nodepart{four}}
\newcommand{\npc}{\nodepart{five}}

\tikzset{fnode/.style={rectangle split, 
  rectangle split part align={left,left,left,center,left}, 
  rectangle split parts=5, draw, minimum width =2.75cm,
  rounded corners}
}

\NewDocumentCommand{\anchormark}{O{0.15 cm} m O{0.05}}{%
  \tikz[overlay,remember picture,baseline=-0.5ex,xshift=#1] 
    \node[draw,fill=black,circle,scale=#3] (#2) {};
}

\tikzset{label style/.style={draw, rounded corners}}

\begin{document}
\begin{tikzpicture}[remember picture]
\node[fnode] (r1) at (0,0) {Puntatore:\anchormark{puntatore 1}[0.075]  
\npd Chiave  \npt Testo \npq 
\anchormark[-0.15cm]{puntatore 2 sinistro}Puntatori 
\anchormark{puntatore 2 destro}\npc Testo};  
\node[fnode] (r2) at (5,0) {Puntatore:\anchormark{puntatore 3}[0.075]  
\npd Chiave  \npt Testo \npq 
\anchormark[-0.05cm]{puntatore 3 sinistro}Puntatori\npc Testo};  
\end{tikzpicture}
\begin{tikzpicture}[remember picture,overlay,-stealth] 
\draw (puntatore 1.center) to ($(puntatore 1.center)+(2,1)$) 
 node[right,label style] (mylabel){label};
\draw (puntatore 2 destro.center) to (puntatore 3 sinistro);
\draw (puntatore 2 sinistro.center) to 
 ($(puntatore 2 sinistro.center)+(-1,1)$)
 node[left,label style]{label};
\draw (puntatore 3.center) |-(mylabel);

% solo per connession esterna dei moduli
\draw(r1.south)|- ($(r1.south)!0.5!(r2.south)-(0,1)$)-|(r2.south);
\end{tikzpicture}
\end{document}
\end{lstlisting}

Il risultato viene mostrato in figura \ref{fig:collegamentifigure}.

\begin{figure}[ht]
\tikzset{document color/.style={draw,top color=orange!5, bottom color=orange!20}}
\tikzset{document size/.style={minimum width=8cm, minimum height=7cm,
 corner size=30pt,
  inner corner size=15pt,
  corner start=45pt,
  corner end=15pt,
  corner up height=15pt,
  corner down height=45pt,
  fine sep start=4.2pt,
  fine sep end=4.2pt,
  }
}
\centering
\includedocument[document color,document size]{sorgcrop/collegamenti-crop}
\caption{Collegamenti fra figure}
\label{fig:collegamentifigure}
\end{figure}

%--------------------------------------------
\section{Annotazioni su formule}

%--------------------------------------------
In questo esempio, tratto da \guidaurl{http://tex.stackexchange.com/questions/75376/overbracket-with-tikz-square-arrow-style}{Overbracket with tikz square arrow style}, l'obbiettivo è selezionare parti di una formula da mettere in relazione ed inserire un'annotazione.

Per la realizzazione è sufficiente la versione semplice della \cs{tikzmark} macro. Ecco il codice completo:
\begin{lstlisting}[frame=lines]
\documentclass{article}
\usepackage{tikz}
\usetikzlibrary{calc}

\usepackage{amsmath}   

\newcommand{\tikzmark}[1]{%
  \tikz[overlay,remember picture] \node (#1) {};
}
% stile per la creazione della freccia 
\tikzset{square arrow/.style={
  to path={-- ++(0,.25) -| (\tikztotarget)}
  }
}

\begin{document}
\[
= -\nu_{gx}  
 \dfrac{\partial \, \left<\tikzmark{a}n\right>}{\partial\,x}  =
 -\nu_{gx} \, 
 \dfrac{\partial \left<\tikzmark{b}\eta_{0}\right>}{\partial \,x} 
 \, \dfrac{\partial \,T}{\partial \,x}
\]
\tikz[overlay,remember picture]{%
 \draw[<->,square arrow, red, thick] 
   ($(a.north)+(0.25em,0.30ex)$) to ($(b.north)+(0.50em,0.30ex)$) ;
 \node [above, yshift=1.50ex, blue] 
   at  ($(a.north)!0.5!(b.north)$) {label};
}
\end{document}
\end{lstlisting}

Dopo avere compilato due volte, il risultato che si otterà è quello mostrato in figura \ref{fig:annotaformula}.

\begin{figure}[ht]
\tikzset{document color/.style={draw,top color=orange!5, bottom color=orange!20}}
\tikzset{document size/.style={minimum width=6.5cm, minimum height=3cm,
 corner size=20pt,
  inner corner size=10pt,
  corner start=30pt,
  corner end=10pt,
  corner up height=10pt,
  corner down height=30pt,
  fine sep start=3pt,
  fine sep end=3pt,
  }
}
\centering
\includedocument[document color,document size]{sorgcrop/selezione_formula-crop}
\caption{Annotazione su formula}
\label{fig:annotaformula}
\end{figure}

%--------------------------------------------
\section{Evidenziare parti del documento}

%--------------------------------------------
L'utilizzo \emph{principe} e \emph{principale} della \cs{tikzmark} macro è la creazione di riquardi atti ad evidenziare parti del documento; con \emph{parti del documento} si intendono parole o parti di frasi, formule, listati di codice ed algoritmi. 

\subsection{Un primo esempio}
In questo primo esempio verranno evidenziati termini di testo con l'ausilio della libreria \library{tikzmark} e di un comando apposito \cs{riquadro}. Tale comando crea semplicemente un rettangolo a partire da una coordinata nota, contrassegnata dalla \cs{tikzmark} macro in precedenza. Si noti che, grazie alla libreria, è possibile utilizzare le coordinate del marcatore con la sintassi \chiavestyle{pic cs:label}, dove \chiavestyle{label} è la label identificativa del marcatore.

Il seguente esempio minimo, con due compilazioni, produce il risultato in figura \ref{fig:evidenziatermini}.
\begin{lstlisting}[frame=lines]
\documentclass{article}
\usepackage{tikz}
\usetikzlibrary{tikzmark}

\newcommand{\riquadro}[1]{%
\tikz[remember picture,overlay]
\draw[line width=1pt,rectangle,rounded corners,
  fill=blue!10,draw=blue]
  (pic cs:#1) ++(0.05,-0.15) rectangle (0.05,0.35);
}

\begin{document}
\riquadro{a}Testo da evidenziare\tikzmark{a} e testo normale.\\[2ex]
Testo normale e \riquadro{a1}testo evidenziato.\tikzmark{a1}\\[2ex]
\end{document}
\end{lstlisting}

\begin{figure}[ht]
\tikzset{document color/.style={draw,top color=orange!5, bottom color=orange!20}}
\tikzset{document size/.style={minimum width=6.5cm, minimum height=3cm,
 corner size=20pt,
  inner corner size=10pt,
  corner start=30pt,
  corner end=10pt,
  corner up height=10pt,
  corner down height=30pt,
  fine sep start=3pt,
  fine sep end=3pt,
  }
}
\centering
\includedocument[document color,document size]{sorgcrop/ev-semplice-crop}
\caption{Evidenziare termini nel testo}
\label{fig:evidenziatermini}
\end{figure}

Ma come funziona esattamente il comando \cs{riquadro}? Analizzando la definizione
\begin{lstlisting}[frame=lines]
\newcommand{\riquadro}[1]{%
\tikz[remember picture,overlay]
\draw[line width=1pt,rectangle,rounded corners,
  fill=blue!10,draw=blue]
  (pic cs:#1) ++(0.05,-0.15) rectangle (0.05,0.35);
}
\end{lstlisting}
si può notare che, a partire dalla coordinata del marcatore \chiavestyle{pic cs:label} se ne aggiorna il valore con \chiavestyle{++(0.05,-0.15)} raggiungendo un punto più in basso e spostato a destra: è uno dei bordi del riquadro. A questo punto si crea un rettangolo verso la coorindata \chiavestyle{(0.05,0.35)} che diventerà chiaramente il bordo opposto al primo del nostro riquadro. Il concetto è spiegato molto bene in figura \ref{fig:estremiriquadro}.

\begin{figure}[ht]
\tikzset{document color/.style={draw,top color=orange!5, bottom color=orange!20}}
\tikzset{document size/.style={minimum width=6.5cm, minimum height=3cm,
 corner size=20pt,
  inner corner size=10pt,
  corner start=30pt,
  corner end=10pt,
  corner up height=10pt,
  corner down height=30pt,
  fine sep start=3pt,
  fine sep end=3pt,
  }
}
\centering
\includedocument[document color,document size]{sorgcrop/semplice-markers-crop}
\caption{Estremi del riquadro}
\label{fig:estremiriquadro}
\end{figure}
L'approccio tuttavia soffre di un problema non da poco: l'altezza del riquadro è infatti fissa. Si ipotizzi di voler evidenziare una frazione:
\begin{lstlisting}[frame=lines]
\documentclass{article}
\usepackage{amsmath,tikz}
\usetikzlibrary{tikzmark}

\newcommand{\riquadro}[1]{%
\tikz[remember picture,overlay]
\draw[line width=1pt,rectangle,rounded corners,
  fill=blue!10,draw=blue]
  (pic cs:#1) ++(0.05,-0.15) rectangle (0.05,0.35);
}

\begin{document}
\[x=\riquadro{a}\dfrac{10}{67}\tikzmark{a}\]
\end{document}
\end{lstlisting}
Il risultato che si otterrebbe è mostrato in figura \ref{fig:problema} e, ovviamente, non è accettabile.
\begin{figure}[ht]
\tikzset{document color/.style={draw,top color=orange!5, bottom color=orange!20}}
\tikzset{document size/.style={minimum width=2.5cm, minimum height=1.5cm,
 corner size=10pt,
  inner corner size=5pt,
  corner start=15pt,
  corner end=5pt,
  corner up height=5pt,
  corner down height=15pt,
  fine sep start=1.3pt,
  fine sep end=1.3pt,
  }
}
\centering
\includedocument[document color,document size]{sorgcrop/problema-crop}
\caption{Riquadro con dimensioni errate}
\label{fig:problema}
\end{figure}

\subsection{Soluzioni più avanzate}

Per ovviare all'inconveniente descritto esistono almeno due possibili strade: la prima suggerita da \href{http://tex.stackexchange.com/users/4301/peter-grill}{Peter Grill} consiste nel determinare le esatte dimensioni del riquadro in modo automatico (riferimenti: \guidaurl{http://tex.stackexchange.com/questions/35319/a-boxed-alternative-with-minimal-spacing/\#35357}{A \textbackslash boxed alternative with minimal spacing?} e \guidaurl{http://tex.stackexchange.com/questions/35217/a-boxed-alternative-with-nicer-spacing/\#35227}{A \textbackslash boxed alternative with nicer spacing?}).

Ad esempio, il seguente codice basato su questo tipo di approccio permette di ottenere il risultato in figura \ref{fig:peter}.
\begin{lstlisting}[frame=lines]
\documentclass{article}
\usepackage{amsmath,tikz}
\usetikzlibrary{tikzmark}

\usetikzlibrary{calc}

\makeatletter
\newcommand*{\@DrawBoxHeightSep}{0.1em}%
\newcommand*{\@DrawBoxDepthSep}{0.08em}%
\newcommand{\@DrawBox}[6][blue]{
%#1= stile, #2=altezza, #3=profondità, #4 marcatore sinistro,
% #5 marcatore destro,
\tikz[overlay,remember picture,baseline]
 \draw[fill=#1!10,draw=#1,rounded corners]
  ($(pic cs:#4)+(-0.2em,#2+\@DrawBoxHeightSep)$) rectangle
  ($(pic cs:#5)+(0.2em,-#3-+\@DrawBoxDepthSep)$);
\tikz[overlay,remember picture,baseline]
 \node[anchor=base] at ($(pic cs:#4)!0.5!(pic cs:#5)$) {#6};
}


\newcounter{image}
\setcounter{image}{1}
\newdimen\@myBoxHeight%
\newdimen\@myBoxDepth%
\newcommand{\MyBox}[2][blue]{%
 \settoheight{\@myBoxHeight}{#2}% Altezza riquadro
 \settodepth{\@myBoxDepth}{#2}% Profondità riquadro
 \tikzmark{l\theimage}#2\tikzmark{r\theimage}
\@DrawBox[#1]{\@myBoxHeight}{\@myBoxDepth}
  {l\theimage}{r\theimage}{#2}
 \stepcounter{image}
}
\makeatother

\begin{document}
\[x=\MyBox{\ensuremath{\dfrac{10}{67}+c}}\]
\end{document}
\end{lstlisting}

\begin{figure}[ht]
\tikzset{document color/.style={draw,top color=orange!5, bottom color=orange!20}}
\tikzset{document size/.style={minimum width=2.5cm, minimum height=1.5cm,
 corner size=10pt,
  inner corner size=5pt,
  corner start=15pt,
  corner end=5pt,
  corner up height=5pt,
  corner down height=15pt,
  fine sep start=1.3pt,
  fine sep end=1.3pt,
  }
}
\centering
\includedocument[document color,document size]{sorgcrop/peter-crop}
\caption{Riquadro con dimensioni create automaticamente}
\label{fig:peter}
\end{figure}
Il metodo ha indubbiamente un grande vantaggio: l'utente finale deve soltanto usare il comando \cs{MyBox} senza preoccuparsi di altro. Tuttavia al fine del calcolo delle dimensioni del riquadro è indispensabile che la zona da evidenziare sia passata come argomento al comando. Questo requisito non è di poco conto in quanto se la zona da evidenziare è su più righe l'approccio non funziona.

La seconda soluzione proposta prevede che l'utente abbia la possibilità di modificare le coordinate che definiscono il bordo del riquadro in modo autonomo. Il riquadro, quindi, può essere \emph{esteso} a piacere: per questo motivo in questo approccio i marcatori sono definiti \emph{estensibili}. È il principio usato dal pacchetto \guidaurl{http://www.ctan.org/pkg/hf-tikz}{\packstyle{hf-tizk}}, disponibile su \TeX Live e MiK\TeX{}.

Come funziona? Ecco un esempio minimo.
\begin{lstlisting}[frame=lines]
\documentclass{article}
\usepackage{amsmath}
\usepackage[customcolors]{hf-tikz}
\hfsetfillcolor{blue!10}
\hfsetbordercolor{blue}

\begin{document}
\[x= \tikzmarkin{a}(0.1,-0.4)(-0.05,0.6) 
\dfrac{10}{67}+c \tikzmarkend{a} 
\hspace*{1cm}
\begin{pmatrix}
1 & \tikzmarkin{a1}0 & 1\\
0 & 1 & 1\\
1 & 0 \tikzmarkend{a1} & 1
\end{pmatrix}
\]
\end{document}
\end{lstlisting}

Dal risultato mostrato in figura \ref{fig:hf-tikz} si può osservare a sinistra la stessa parte di formula già evidenziata nell'esempio precedente; questa volta, però, si noti che occorre passare al comando \cs{tikzmarkin} sia la label che identifica il marcatore che due coordinate: rappresentano proprio il bordo inferiore destro e superiore sinistro del riquadro che sono stati estesi rispetto ai valori di default.

\begin{figure}[ht]
\tikzset{document color/.style={draw,top color=orange!5, bottom color=orange!20}}
\tikzset{document size/.style={minimum width=5.75cm, minimum height=2.75cm,
 corner size=20pt,
  inner corner size=10pt,
  corner start=30pt,
  corner end=10pt,
  corner up height=10pt,
  corner down height=30pt,
  fine sep start=3pt,
  fine sep end=3pt,
  }
}
\centering
\includedocument[document color,document size]{sorgcrop/hf-tikz-crop}
\caption{Riquadro con dimensioni definite dall'utente}
\label{fig:hf-tikz}
\end{figure}

L'approccio inoltre permette facilmente anche di evidenziare parti di matrici che, essendo su più righe, il metodo precedente non riusciva a fare.

Cosa conviene usare dunque? Il primo metodo funziona certamente meglio quando gli elementi da evidenziare sono su singole righe del documento; in caso di formule è necessario racchiudere in \cs{ensuremath\{\}} la parte di formula da evidenziare altrimenti si incorre in un errore. Con il metodo dei marcatori estensibili non esistono queste limitazioni, ma l'utente deve avere pazienza e fare qualche tentativo per \emph{trovare} le coordinate giuste.

\section{Evidenziare ed inserire un'annotazione}
Si prende ora in esame un esempio decisamente più complesso: come si è visto, per evidenziare parti del documento esistono diversi metodi rodati; ora si richiede in più la possibilità di inserire un'annotazione da affiancare alla parte di testo evidenziata. Nell'esempio, tratto da \guidaurl{http://tex.stackexchange.com/questions/57060/mark-a-pseudocode-block-and-insert-comments-near-it}{Mark a pseudocode block and insert comments near it}, si vuole evidenziare parte di un algoritmo.

Il comando principale, \cs{riquadro}, questa volta è capace di supportare i marcatori estensibili e, poiché si tratta di una presentazione, come primo argomento richiede in quali istanti di tempo sarà visibile il riquadro.

Per inserire l'annotazione si sfrutta la proprietà per cui grazie alla sintassi \chiavestyle{pic cs:label} si accede alle coordinate del marcatore \chiavestyle{label}: si definità una \emph{coordinata base} a partire da quella del marcatore e si inserirà l'annotazione come un semplice nodo di \Tikz{}.

L'esempio completo:
\begin{lstlisting}[frame=lines]
\documentclass{beamer}
\usepackage{algorithm,algpseudocode}
\usepackage{xparse}
\usepackage{tikz}
\usetikzlibrary{calc,tikzmark}

\NewDocumentCommand{\riquadro}{r<> m D(){5.2,-.1} D(){-0.4,0}}{%
\only<#1>{\tikz[remember picture with id=#2]
\draw[line width=1pt,fill=blue!10, draw=blue,rectangle,rounded corners]
node [anchor=base] (#2){} (pic cs:#2) ++(#3) rectangle (#4);
	}
}

\begin{document}
\begin{frame}
\begin{algorithm}[H]
\begin{algorithmic}
 \Function{tarjan}{Node* node}        
  \State $node.visited \gets $ \textbf{true}
  \State $node.index \gets indexCounter$
  \State $s.push(node)$
  \ForAll{$successor$ in $node.successors$}
    \If{$!node.visited$}
         \Call{tarjan}{successor}
   \EndIf
   \State $node.lowlink \gets$ \Call{min}{$node.lowlink, successor.lowlink$}
  \EndFor
            
  \riquadro<1->{a1}\If{$node.lowlink == node.index$}
    \Repeat 
       \State $successor \gets stack.pop()$
       \Until{$successor == node$}
     \EndIf\tikzmark{a1}
 \EndFunction
\end{algorithmic}
\label{alg:seq2}
\end{algorithm}

% Inserire l'annotazione
\tikzset{annotazione/.style={%
		rectangle,draw, gray,text width=3cm,
		align=justify,right
	}
}
\begin{tikzpicture}[remember picture,overlay]
% modificare questo parametro per cambiare 
% la coordinata alla base dell'annotazione
\coordinate (annota) at ($(pic cs:a1)+(5.5,1.25)$); 
\node[annotazione] at (annota) 
 {Annotazione da inserire accanto alla parte evidenziata};
\end{tikzpicture}
\end{frame}

\end{document}
\end{lstlisting}

Il risultato ottenuto viene mostrato in figura \ref{fig:evidenziaannotazione}.

\begin{figure}[ht]
\tikzset{document color/.style={draw,top color=orange!5, bottom color=orange!20}}
\tikzset{document size/.style={minimum width=10cm, minimum height=8.5cm,
 corner size=30pt,
  inner corner size=15pt,
  corner start=45pt,
  corner end=15pt,
  corner up height=15pt,
  corner down height=45pt,
  fine sep start=4.4pt,
  fine sep end=4.4pt,
  }
}
\centering
\includedocument[document color,document size][scale=0.65]{sorgtikz/annota_algoritmo}
\caption{Evidenziare ad annotare alcuni passaggi di un algoritmo}
\label{fig:evidenziaannotazione}
\end{figure}