\documentclass{beamer}
\usepackage{algorithm,algpseudocode}
\usepackage{xparse}
\usepackage{tikz}
\usetikzlibrary{calc,tikzmark}

\NewDocumentCommand{\riquadro}{r<> m D(){5.2,-.1} D(){-0.4,0}}{%
\only<#1>{\tikz[remember picture with id=#2]
\draw[line width=1pt,fill=blue!10, draw=blue,rectangle,rounded corners]
node [anchor=base] (#2){} (pic cs:#2) ++(#3) rectangle (#4);
	}
}

\begin{document}
\begin{frame}
\begin{algorithm}[H]
\begin{algorithmic}
 \Function{tarjan}{Node* node}        
  \State $node.visited \gets $ \textbf{true}
  \State $node.index \gets indexCounter$
  \State $s.push(node)$
  \ForAll{$successor$ in $node.successors$}
    \If{$!node.visited$}
         \Call{tarjan}{successor}
   \EndIf
   \State $node.lowlink \gets$ \Call{min}{$node.lowlink, successor.lowlink$}
 \EndFor
            
   \riquadro<1->{a1}\If{$node.lowlink == node.index$}
     \Repeat 
        \State $successor \gets stack.pop()$
         \Until{$successor == node$}
      \EndIf\tikzmark{a1}
   \EndFunction
\end{algorithmic}
\label{alg:seq2}
\end{algorithm}

% To insert the annotation
\begin{tikzpicture}[remember picture,overlay]
\tikzset{annotazione/.style={%
		rectangle,draw, gray,text width=3cm,
		align=justify,right
	}
}
\coordinate (annota) at ($(pic cs:a1)+(5.5,1.25)$); 
% modificare questo parametro per cambiare 
% la coordinata ancora dell'annotazione
\node[annotazione] at (annota) 
 {Annotazione da inserire accanto alla parte evidenziata};
\end{tikzpicture}
\end{frame}

\end{document}